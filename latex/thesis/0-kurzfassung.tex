\chapter*{Kurzfassung}
\addcontentsline{toc}{chapter}{Kurzfassung}

Die von \textcite{LTH} aufgestellte Lottery Ticket Hypothese hat einen neuen Teilbereich der Forschung begründet, der sich mit \textit{Sparse Neural Networks} beschäftigt, welche erfolgreich trainiert werden könne.
\textit{Iterative Magnitude Pruning} ist eine Schlüsselmethode für dieses Feld, da sie die schlanken und performanten Subnetzwerke, die sogenannten \textit{Winning Tickets} erfolgreich ausfindig machen kann.
Wieso diese Methode erfolgreich ist und was die Subnetzwerke, die \textit{Winning Tickets} so erfolgreich macht, ist großteils unerklärt.
Eine Möglichkeit um \textit{Winning Tickets} und deren Erzeugung besser zu verstehen, ist die Analyse des Netzwerkgraphens.
Dadurch, dass diese Subnetze \textit{sparse} sind, könnte ihre Struktur wertvolle Einblicke in die Funktion des Netzwerkes bieten.
Im Rahmen dieser Masterarbeit wird die Struktur der \textit{Winning Tickets} und ihre Verbindung zur Funktion des Netzes untersucht.
Dazu wurden Experimente auf Datensätzen mit unabhängigen Aufgaben erstellt \rule[0.5ex]{.5em}{0.5pt} ein einfacher synthetischer Datensatz und eine Kombination aus dem MNIST und dem Fashion-MNIST Datensatzes.
Mittels \textit{Iterative Magnitude Pruning} können für beide Datensätze unabhängige Subnetwzerke im \textit{Winning Ticket} gefunden werden, wobei jedes Subnetzwerk eine Aufgabe des Datensatzes löst.