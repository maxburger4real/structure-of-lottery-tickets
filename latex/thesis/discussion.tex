\chapter{Conslusion}\label{chapter:conclusion} 

Over different network sizes, different architectures with varying numbers of hidden layers as well as different pruning rates, the networks separate quite reliably, given the right conditions.
The experiments where the network did not separate either had too small networks, too few pruning levels and too high of a pruning rate.
Also for the more complicated MNIST-Fashion-MNIST dataset, the networks separated with the standard training iterative pruning method.

The results indicate, that iterative magnitude pruning indeed produces lottery tickets with separate subnetworks for independent tasks.

The lottery ticket hypothesis sparked many directions of research in the field of sparse network training and pruning in general.
However, how and why they work remains elusive.

In this thesis, an empirical analysis was conducted on the relationship between the semantic structure of the data and the structure of the sparse lottery ticket subnetwork.
In the experiments conducted, it was shown that networks trained with \textit{Iterative Magnitude Pruning}, produced two independent subnetworks on datasets with two independent tasks.
Systematic experiments were conducted on the `Moons-Circles' dataset, a simplistic toy dataset that contains two independent tasks.
Due to the success of the method on the toy task, experiments were also conducted on the `MNIST-Fashion-MNIST' dataset, which is significantly more realistic.
In both cases, neural networks were found to separate with non-trivial accuracy.

\section{Limitations}
The findings of this thesis are empirical, and due the the scope and scale of the thesis, there are numerous untested scenarios.
The experiments were only conducted on toy datasets and on MNIST-Fashion-MNIST which might also be counted as a toy dataset.
Whether the findings remain true for larger and more realistic tasks remains an open question.

\section{Discussion}
The experiments in this thesis demonstrate that iterative magnitude pruning finds independent subnetworks.
This can be valuable for understanding what the structure of lottery tickets represents.
Further, it could also be valuable for machine learning interpretability.
A lottery ticket could be trained and if independent networks occur, it could hint at independence in the dataset.
The same might be true for other types of semantic structure, like feature sharing or compositionality.

\section{Future Work}
An interesting avenue for future research in this line of work would be to examine the numerous other variants for finding sparse subnetworks.
Would they also separate? And if so, would they separate earlier or with better performance still?
Also, an investigation into the influence of things that worked well for lottery tickets, like learning rate schedules and resetting, as well as L1 or L2-regularisation.
Similar to the work in \autocite{BIMT}, other semantic structures of the data might be considered, like compositionality or weight sharing.
Finally, examine the methods on larger and more realistic datasets like CIFAR \autocite{cifar} or ImageNet \autocite{imagenet}, as well as translating the findings into convolutional neural net architectures.