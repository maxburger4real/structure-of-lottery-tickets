\chapter*{Kurzfassung}
\addcontentsline{toc}{chapter}{Kurzfassung}

Die von \textcite{LTH} aufgestellte \textbf{Lottery Ticket Hypothese} hat einen neuen Teilbereich der Forschung begründet, der sich mit `Sparse Neural Networks' beschäftigt.
`Iterative Magnitude Pruning' (IMP) ist eine Schlüsselmethode für dieses Feld, da sie die schlanken und performanten Subnetzwerke, die sogenannten \textit{Winning Tickets} erfolgreich ausfindig machen kann.
Jedoch sind viele Aspekte der Funktionsweise dieser Technik noch unerklärt.
Eine Möglichkeit um \textit{Winning Tickets} und deren Erzeugung besser zu verstehen, ist die Analyse des Netzwerkgraphens.
Dadurch, dass \textit{Winning Tickets} `sparse' sind, könnte ihre Struktur wertvolle Einblicke in die Funktion des Netzwerkes bieten.
Um mehr über die Struktur dieser Netze zu erfahren wurden Experimente durchgeführt, welche von \textcite{BIMT} inspiriert wurden.
Ein Datensatz, welcher zwei voneinander unabhängige Aufgaben enthält, wurde genutzt, um mittels \textit{IMP} Subnetze zu finden.
Folgende Frage wird in dieser Arbeit gestellt:
Enthält das \textit{Winning Ticket} zwei unabhängige Subnetzte, eines für jede Aufgabe?