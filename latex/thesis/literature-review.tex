\section{Literature Review}


\subsection{introduction}
at the core this thesis is about understanding the structure of lottery tickets. So what is or was important about that?
\begin{itemize}
    \item deep learning, overparameterization and pruning
    \item lottery ticket Hypothesis, its relevance 
    \item derivative works of the lth, attempts to understand it
    \item structure of lottery tickets ? is there really nothing? connectedness, degree (original LTH), Han et al says there will be dead neurons, All allive pruning says they kill them
    \item understanding networks, mechanistic interpretability, independence as a nice thing to understand stuff. Bimt showing independece is in the network
\end{itemize}

Is the structure of the lottery ticket useful? interpretable? or is it merely super overfitted random shit that doesnt help you anyway.

\subsection{Deep Learning}
\subsection{Neural Network Pruning}
\subsection{The Lottery Ticket Hypothesis}
empirical 
\subsubsection{Pruning Strategies}
\subsubsection{Weight Rewinding}
\subsubsection{Theoretical Understanding}

\subsection{Finding Lottery Tickets}
several algorithms have been proposed to find lottery tickets. here is an inexaustive collection of some of them 
\subsubsection{Supermasks}
\subsubsection{Edge Popup}
\subsubsection{Rare Gems}

\subsection{Understanding Lottery Tickets}
find attempts where people tried to understand
\subsection{All Alive Pruning}
acknoledges dead connections
\subsubsection{Mechanistic Interpretability}
is the basis of understanding the networks
\subsubsection{symbolic regression, BIMT}
inspired the experiments




\subsection{Other possibly related stuff}
\begin{itemize}
    \item dynamic sparse training
    \item network distillation
    \item tinyML
\end{itemize}



\subsection{about}
A literature review helps you gain a robust understanding of any extant academic work on your topic, encompassing:
\begin{itemize}
    \item Selecting relevant sources
    \item Determining the credibility of your sources
    \item Critically evaluating each of your sources
    \item Drawing connections between sources, including any themes, patterns, conflicts, or gaps
  \end{itemize}


A literature review is not merely a summary of existing work. Rather, your literature review should ultimately lead to a clear justification for your own research, perhaps via:
\begin{itemize}
    \item Addressing a gap in the literature
    \item Building on existing knowledge to draw new conclusions
    \item Exploring a new theoretical or methodological approach
    \item Introducing a new solution to an unresolved problem
    \item Definitively advocating for one side of a theoretical debate
\end{itemize}
